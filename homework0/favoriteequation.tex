\documentclass [11pt]{article}
\usepackage[utf8]{inputenc}
\usepackage{graphicx}
\usepackage{subcaption}
\title{My favorite equation}
\author{Shaun Fedrick}
\begin{document}
\maketitle
\begin{math}\frac{df(x)}{dx}= lim_{h \rightarrow 0} \frac{f(x+h)-f(x)}{h}\end{math}
\\
\par{
This is the definition of the derivative. I think this is my favorite equation because I feel that this equation is at the core of most of the math I do day to day. 
}
\end{document}